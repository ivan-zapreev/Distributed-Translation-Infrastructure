\#$\ast$$\ast$\+The Basic Phrase-\/\+Based Statistical Machine Translation Tool$\ast$$\ast$

{\bfseries Author\+:} \href{https://nl.linkedin.com/in/zapreevis}{\tt Dr. Ivan S. Zapreev}

{\bfseries Project pages\+:} \href{https://github.com/ivan-zapreev/Back-Off-Language-Model-SMT}{\tt Git-\/\+Hub-\/\+Project}

\subsection*{Introduction}

This is a fork project from the Back Off Language Model(s) for S\+M\+T project aimed at creating the entire phrase-\/based S\+M\+T translation infrastructure. This project follows a client/server atchitecture based on Web\+Sockets for C++ and consists of the three main applications\+:


\begin{DoxyItemize}
\item {\bfseries bpbd-\/client} -\/ is a thin client to send the translation job requests to the translation server and obtain results
\item {\bfseries bpbd-\/server} -\/ the the translation server consisting of the following main components\+:
\begin{DoxyItemize}
\item {\itshape Decoder} -\/ the decoder component responsible for translating text from one language into another
\item {\itshape L\+M} -\/ the language model implementation allowing for seven different trie implementations and responsible for estimating the target language phrase probabilities.
\item {\itshape T\+M} -\/ the translation model implementation required for providing source to target language phrase translation and the probailities thereof.
\item {\itshape R\+M} -\/ the reordering model implementation required for providing the possible translation order changes and the probabilities thereof
\end{DoxyItemize}
\item {\bfseries lm-\/query} -\/ a stand-\/alone language model query tool that allows to perform labguage model queries and estimate the joint phrase probabilities.
\end{DoxyItemize}

To keep a clear view of the used terminology further we will privide some details on the phrase based statistical machine translation as given on the picture below.



The entire phrase-\/based statistical machine translation is based on learned statistical correlations between words and phrases of an example translation text, also called parallel corpus or corpora. Clearly, if the training corpora is large enough then it allows to cover most source/target language words and phrases and shall have enough information for approximating a translation of an arbitrary text. However, before this information can be extracted, the parallel corpora undergoes the process called {\itshape word alignment} which is aimed at estimating which words/phrases in the source language correspond to which words/phrases in the target language. As a result, we obtain two statistical models\+:


\begin{DoxyEnumerate}
\item The Translation model -\/ providing phrases in the source language with learned possible target language translations and the probabilities thereof.
\item The Reordering model -\/ storing information about probable translation orders of the phrases within the source text, based on the observed source and target phrases and alignment thereof.
\end{DoxyEnumerate}

The last model, possibly learned from a different corpus in a target language, is the Language model. Its purpose is to reflect the likelihood of this or that phrase in the target language to occur. In other words it is used to evaluate the obtained translation for being {\itshape sound} in the target language.

With these three models at hand one can perform decoding, which is a synonim to a translation process. S\+M\+T decoding is performed by exploring the state space of all possible translations and reorderings of the source language phrases within one sentence and then looking for the most probable translations, as indicated at the bottom part of the picture above.

The rest of the document is organized as follows\+:


\begin{DoxyEnumerate}
\item \href{#project-structure}{\tt Project structure} -\/ Gives the file and folder structure of the project
\item \href{#supported-platforms}{\tt Supported platforms} -\/ Indicates the project supported platforms
\item \href{#building-the-project}{\tt Building the project} -\/ Describes the process of building the project
\item \href{#using-software}{\tt Using software} -\/ Explain how the software is to be used
\item \href{#input-file-formats}{\tt Input file formats} -\/ Provides examples of the input file formats
\item \href{#code-documentation}{\tt Code documentation} -\/ Refers to the project documentation
\item \href{#external-libraries}{\tt External libraries} -\/ Lists the included external libraries
\item \href{#general-design}{\tt General design} -\/ Outlines the general software desing
\item \href{#software-details}{\tt Software details} -\/ Goes about some of the software details
\item \href{#literature-and-references}{\tt Literature and references} -\/ Presents the list of used literature
\item \href{#licensing}{\tt Licensing} -\/ States the licensing strategy of the project
\item \href{#history}{\tt History} -\/ Stores a short history of this document
\end{DoxyEnumerate}

\subsection*{Project structure}

This is a Netbeans 8.\+0.\+2 project, based on cmake, and its top-\/level structure is as follows\+:


\begin{DoxyItemize}
\item $\ast$$\ast${\ttfamily \mbox{[}Project-\/\+Folder\mbox{]}}$\ast$$\ast$/
\begin{DoxyItemize}
\item {\bfseries doc/} -\/ contains the project-\/related documents including the Doxygen-\/generated code documentation and images
\item {\bfseries ext/} -\/ stores the external header only libraries used in the project
\item {\bfseries inc/} -\/ stores the C++ header files of the implementation
\item {\bfseries src/} -\/ stores the C++ source files of the implementation
\item {\bfseries nbproject/} -\/ stores the Netbeans project data, such as makefiles
\item {\bfseries data/} -\/ stores the test-\/related data such as test models and query intput files, as well as some experimental results.
\item L\+I\+C\+E\+N\+S\+E -\/ the code license (G\+P\+L 2.\+0)
\item C\+Make\+Lists.\+txt -\/ the cmake build script for generating the project\textquotesingle{}s make files
\item \hyperlink{_r_e_a_d_m_e_8md}{R\+E\+A\+D\+M\+E.\+md} -\/ this document
\item Doxyfile -\/ the Doxygen configuration file
\end{DoxyItemize}
\end{DoxyItemize}

\subsection*{Supported platforms}

This project supports two major platforms\+: Linux and Mac Os X. It has been successfully build and tested on\+:


\begin{DoxyItemize}
\item {\bfseries Centos 6.\+6 64-\/bit} -\/ Complete functionality.
\item {\bfseries Ubuntu 15.\+04 64-\/bit} -\/ Complete functionality.
\item {\bfseries Mac O\+S X Yosemite 10.\+10 64-\/bit} -\/ Limited by inability to collect memory-\/usage statistics.
\end{DoxyItemize}

{\bfseries Notes\+:}


\begin{DoxyEnumerate}
\item There was only a limited testing performed on 32-\/bit systems.
\item The project must be possible to build on Windows platform under \href{https://www.cygwin.com/}{\tt Cygwin}.
\end{DoxyEnumerate}

\subsection*{Building the project}

Building this project requires {\bfseries gcc} version $>$= {\itshape 4.\+9.\+1} and {\bfseries cmake} version $>$= 2.\+8.\+12.\+2. The project can be build in two ways\+:


\begin{DoxyItemize}
\item From the Netbeans environment by running Build in the I\+D\+E
\begin{DoxyItemize}
\item Perform {\ttfamily mkdir build} in the project folder.
\item In Netbeans menu\+: {\itshape Tools/\+Options/\char`\"{}\+C/\+C++\char`\"{}} make sure that the cmake executable is properly set.
\item Netbeans will always run cmake for the D\+E\+B\+U\+G version of the project
\item To build project in R\+E\+L\+E\+A\+S\+E version use building from Linux console
\end{DoxyItemize}
\item From the Linux command-\/line console perform the following steps
\begin{DoxyItemize}
\item {\ttfamily cd \mbox{[}Project-\/\+Folder\mbox{]}}
\item {\ttfamily mkdir build}
\item {\ttfamily cd build}
\item {\ttfamily cmake -\/\+D\+C\+M\+A\+K\+E\+\_\+\+B\+U\+I\+L\+D\+\_\+\+T\+Y\+P\+E=Release ..} O\+R {\ttfamily cmake -\/\+D\+C\+M\+A\+K\+E\+\_\+\+B\+U\+I\+L\+D\+\_\+\+T\+Y\+P\+E=Debug ..}
\item {\ttfamily make -\/j \mbox{[}N\+U\+M\+B\+E\+R-\/\+O\+F-\/\+T\+H\+R\+E\+A\+D\+S\mbox{]}} add {\ttfamily V\+E\+R\+B\+O\+S\+E=1} to make the compile-\/time options visible
\end{DoxyItemize}
\end{DoxyItemize}

The binaries will be generated and placed into $\ast$./build/$\ast$ folder. In order to clean the project from the command line run {\ttfamily make clean}. Cleaning from Netbeans is as simple calling the {\ttfamily Clean and Build} from the {\ttfamily Run} menu.

\subsubsection*{Project compile-\/time parameters}

There is a number of project parameters that at this moment are to be chosen only once before the project is compiled. These are otherwise called the compile-\/time parameters. Further we consider the most important of them and indicate where all of them are to be found.

{\bfseries Loggin level\+:} Logging is important when debugging software or providing an additional used information during the program\textquotesingle{}s runtime. Yet additional output actions come at a prise and can negatively influence the program\textquotesingle{}s performance. This is why it is important to be able to disable certain logging levels within the program not only during its runtime but also at compile time. The possible range of project\textquotesingle{}s logging levels, listed incrementally is\+: E\+R\+R\+O\+R, W\+A\+R\+N\+I\+N\+G, U\+S\+A\+G\+E, R\+E\+S\+U\+L\+T, I\+N\+F\+O, I\+N\+F\+O1, I\+N\+F\+O2, I\+N\+F\+O3, D\+E\+B\+U\+G, D\+E\+B\+U\+G1, D\+E\+B\+U\+G2, D\+E\+B\+U\+G3, D\+E\+B\+U\+G4. One can limit the logging level range available at runtime by setting the {\ttfamily L\+O\+G\+E\+R\+\_\+\+M\+\_\+\+G\+R\+A\+M\+\_\+\+L\+E\+V\+E\+L\+\_\+\+M\+A\+X} constaint value in the {\ttfamily ./inc/common/utils/logging/logger.hpp} header file.

{\bfseries Sanity checks\+:} When program is not running as expected, it could be caused by the internal software errors that are not detectable runtime. It is therefore possible to enable/disable software internal sanity checks by setting the {\ttfamily D\+O\+\_\+\+S\+A\+N\+I\+T\+Y\+\_\+\+C\+H\+E\+C\+K\+S} constand in the {\ttfamily ./inc/common/utils/exceptions.hpp} header file. Note that enabling the sanity checks does not guarantee that the internal error will be found and will have a negative effect on the program\textquotesingle{}s performance. Yet, it might help to identify errors with e.\+g. input file formats and alike.

{\bfseries Server configs\+:} There is a number of translation server common parameters used in decoding, translation, reordering anb language models. Those are to be found in the {\ttfamily ./inc/server/server\+\_\+configs.hpp}. Please be carefull changing them\+:


\begin{DoxyItemize}
\item {\ttfamily U\+N\+K\+N\+O\+W\+N\+\_\+\+L\+O\+G\+\_\+\+P\+R\+O\+B\+\_\+\+W\+E\+I\+G\+H\+T} -\/ The value used for the unknown probability weight \+\_\+(log10 scale)\+\_\+
\item {\ttfamily Z\+E\+R\+O\+\_\+\+L\+O\+G\+\_\+\+P\+R\+O\+B\+\_\+\+W\+E\+I\+G\+H\+T} -\/ The value used for the \textquotesingle{}zero\textquotesingle{} probability weight \+\_\+(log10 scale)\+\_\+
\item {\ttfamily tm\+::\+N\+U\+M\+\_\+\+T\+M\+\_\+\+F\+E\+A\+T\+U\+R\+E\+S} -\/ The number of the translation model features, defines the number of features read per entry in from the translation model input file.
\item {\ttfamily tm\+::\+T\+M\+\_\+\+M\+A\+X\+\_\+\+T\+A\+R\+G\+E\+T\+\_\+\+P\+H\+R\+A\+S\+E\+\_\+\+L\+E\+N} -\/ The maximum length of the target phrase to be considered, this defines the maximum number of tokens to be stored per translation entry
\item {\ttfamily lm\+::\+N\+U\+M\+\_\+\+L\+M\+\_\+\+F\+E\+A\+T\+U\+R\+E\+S} -\/ The number of languahe model features, the program currenly supports only one value\+: {\ttfamily 1}
\item {\ttfamily lm\+::\+L\+M\+\_\+\+M\+\_\+\+G\+R\+A\+M\+\_\+\+L\+E\+V\+E\+L\+\_\+\+M\+A\+X} -\/ The languahe model maximum level, the maximum number of words in the language model phrase
\item {\ttfamily lm\+::\+L\+M\+\_\+\+H\+I\+S\+T\+O\+R\+Y\+\_\+\+L\+E\+N\+\_\+\+M\+A\+X} -\/ {\bfseries do not change} this parameter
\item {\ttfamily lm\+::\+L\+M\+\_\+\+M\+A\+X\+\_\+\+Q\+U\+E\+R\+Y\+\_\+\+L\+E\+N} -\/ {\bfseries do not change} this parameter
\item {\ttfamily lm\+::\+D\+E\+F\+\_\+\+U\+N\+K\+\_\+\+W\+O\+R\+D\+\_\+\+L\+O\+G\+\_\+\+P\+R\+O\+B\+\_\+\+W\+E\+I\+G\+H\+T} -\/ The default unknown word probability weight, for the case the {\ttfamily $<$unk$>$} entry is not present in the language model file \+\_\+(log10 scale)\+\_\+
\item {\ttfamily rm\+::\+N\+U\+M\+\_\+\+R\+M\+\_\+\+F\+E\+A\+T\+U\+R\+E\+S} -\/ The maximum number of reordering model features, the only two currently supported values are\+: {\ttfamily 6} and {\ttfamily 8}.
\end{DoxyItemize}

{\bfseries Decoder configs\+:} There is a number of decoder-\/specific parameters that can be configured runtime. These are located in {\ttfamily ./inc/server/decoder/de\+\_\+configs.hpp}, please be careful changing them\+:


\begin{DoxyItemize}
\item {\ttfamily M\+A\+X\+\_\+\+W\+O\+R\+D\+S\+\_\+\+P\+E\+R\+\_\+\+S\+E\+N\+T\+E\+N\+C\+E} -\/ The maximum allowed number of words/tokens per sentence to translate.
\end{DoxyItemize}

{\bfseries L\+M configs\+:} There is a number of Language-\/model-\/specific parameters that can be configured runtime. These are located in {\ttfamily ./inc/server/lm/lm\+\_\+configs.hpp}, please be careful changing them\+:


\begin{DoxyItemize}
\item {\ttfamily lm\+\_\+word\+\_\+index} -\/ the word index type to be used, the possible values are\+:
\begin{DoxyItemize}
\item {\ttfamily basic\+\_\+word\+\_\+index} -\/ {\itshape To\+Do\+: Fill In}
\item {\ttfamily counting\+\_\+word\+\_\+index} -\/ {\itshape To\+Do\+: Fill In}
\item {\ttfamily optimizing\+\_\+word\+\_\+index$<$basic\+\_\+word\+\_\+index$>$} -\/ {\itshape To\+Do\+: Fill In}
\item {\ttfamily optimizing\+\_\+word\+\_\+index$<$counting\+\_\+word\+\_\+index$>$} -\/ {\itshape To\+Do\+: Fill In}
\item {\ttfamily hashing\+\_\+word\+\_\+index} -\/
\end{DoxyItemize}
\item {\ttfamily lm\+\_\+model\+\_\+type} -\/ the model type to be used, the possible values (trie types) are\+:
\begin{DoxyItemize}
\item {\ttfamily c2d\+\_\+hybrid\+\_\+trie$<$lm\+\_\+word\+\_\+index$>$} -\/ {\itshape To\+Do\+: Fill In}
\item {\ttfamily c2d\+\_\+map\+\_\+trie$<$lm\+\_\+word\+\_\+index$>$} -\/ {\itshape To\+Do\+: Fill In}
\item {\ttfamily c2w\+\_\+array\+\_\+trie$<$lm\+\_\+word\+\_\+index$>$} -\/ {\itshape To\+Do\+: Fill In}
\item {\ttfamily g2d\+\_\+map\+\_\+trie$<$lm\+\_\+word\+\_\+index$>$} -\/ {\itshape To\+Do\+: Fill In}
\item {\ttfamily h2d\+\_\+map\+\_\+trie$<$lm\+\_\+word\+\_\+index$>$} -\/ {\itshape To\+Do\+: Fill In}
\item {\ttfamily w2c\+\_\+array\+\_\+trie$<$lm\+\_\+word\+\_\+index$>$} -\/ {\itshape To\+Do\+: Fill In}
\item {\ttfamily w2c\+\_\+hybrid\+\_\+trie$<$lm\+\_\+word\+\_\+index$>$} -\/ {\itshape To\+Do\+: Fill In}
\end{DoxyItemize}
\item {\ttfamily lm\+\_\+model\+\_\+reader} -\/ the model reader is basically the file reader type one can use to load the model, currently there are three model reader types available, with {\ttfamily cstyle\+\_\+file\+\_\+reader} being the default\+:
\begin{DoxyItemize}
\item {\ttfamily file\+\_\+stream\+\_\+reader} -\/ uses the C++ streams to read from files, the slowest
\item {\ttfamily cstyle\+\_\+file\+\_\+reader} -\/ uses C-\/style file reading functions, faster than {\ttfamily file\+\_\+stream\+\_\+reader}
\item {\ttfamily memory\+\_\+mapped\+\_\+file\+\_\+reader} -\/ uses memory-\/mapped files which, faster than {\ttfamily cstyle\+\_\+file\+\_\+reader}, consumes twise the file size memory (virtual R\+A\+M).
\end{DoxyItemize}
\item {\ttfamily lm\+\_\+builder\+\_\+type} -\/ currently there is just one builder type available\+: {\ttfamily lm\+\_\+basic\+\_\+builder$<$lm\+\_\+model\+\_\+reader$>$}.
\end{DoxyItemize}

Note that not all of the combinations of the {\ttfamily lm\+\_\+word\+\_\+index} and {\ttfamily lm\+\_\+model\+\_\+type} can work together, this is reported runtime after the program is build. Some additional details on the preferred configurations can be also found in the {\ttfamily ./inc/server/lm/lm\+\_\+consts.hpp} header file comments. The default and the most optimal performance/memory ratio configuration is {\ttfamily lm\+\_\+word\+\_\+index} being set to {\ttfamily hashing\+\_\+word\+\_\+index} and {\ttfamily lm\+\_\+model\+\_\+type} begin set to {\ttfamily h2d\+\_\+map\+\_\+trie$<$lm\+\_\+word\+\_\+index$>$}.

{\bfseries T\+M configs\+:} There is a number of Translation-\/model-\/specific parameters that can be configured runtime. These are located in {\ttfamily ./inc/server/tm/tm\+\_\+configs.hpp}, please be careful changing them\+:


\begin{DoxyItemize}
\item {\ttfamily tm\+\_\+model\+\_\+type} -\/ currently there is just one model type available\+: {\ttfamily tm\+\_\+basic\+\_\+model}.
\item {\ttfamily tm\+\_\+model\+\_\+reader} -\/ the same as {\ttfamily lm\+\_\+model\+\_\+reader} for \+\_\+\char`\"{}\+L\+M configs\char`\"{}\+\_\+ above.
\item {\ttfamily tm\+\_\+builder\+\_\+type} -\/ currently there is just one builder byte available\+: {\ttfamily tm\+\_\+basic\+\_\+builder$<$tm\+\_\+model\+\_\+reader$>$}.
\end{DoxyItemize}

{\bfseries R\+M configs\+:} There is a number of Reordering-\/model-\/specific parameters that can be configured runtime. These are located in {\ttfamily ./inc/server/rm/rm\+\_\+configs.hpp}, please be careful changing them\+:


\begin{DoxyItemize}
\item {\ttfamily rm\+\_\+model\+\_\+type} -\/ currently there is just one model type available\+: {\ttfamily rm\+\_\+basic\+\_\+model}.
\item {\ttfamily rm\+\_\+model\+\_\+reader} -\/ the same as {\ttfamily lm\+\_\+model\+\_\+reader} for \+\_\+\char`\"{}\+L\+M configs\char`\"{}\+\_\+ above.
\item {\ttfamily rm\+\_\+builder\+\_\+type} -\/ currently there is just one builder byte available\+: {\ttfamily rm\+\_\+basic\+\_\+builder$<$rm\+\_\+model\+\_\+reader$>$}.
\end{DoxyItemize}

\subsection*{Using software}

\subsubsection*{\+\_\+bpbd-\/server\+\_\+ -\/ translation server}

{\itshape To\+Do\+: server console}

{\itshape To\+Do\+: Configuration file} \subsubsection*{\+\_\+bpbd-\/client\+\_\+ -\/ translation client}

\subsubsection*{\+\_\+lm-\/query\+\_\+ -\/ language model query tool}

In order to get the program usage information please run $\ast$./lm-\/query$\ast$ from the command line, the output of the program is supposed to be as follows\+:

``` vpn-\/stud-\/146-\/50-\/150-\/5\+:build zapreevis\$ lm-\/query U\+S\+A\+G\+E\+: -\/-\/-\/-\/-\/-\/-\/-\/-\/-\/-\/-\/-\/-\/-\/-\/-\/-\/-\/-\/-\/-\/-\/-\/-\/-\/-\/-\/-\/-\/-\/-\/-\/-\/-\/-\/-\/-\/-\/-\/-\/-\/-\/-\/-\/-\/-\/-\/-\/-\/-\/-\/-\/-\/-\/-\/-\/-\/-\/-\/-\/-\/-\/--- U\+S\+A\+G\+E\+: $\vert$ Back Off Language Model(s) for S\+M\+T \+:)\+\_\+\+\_\+\+\_\+/(\+: $\vert$ U\+S\+A\+G\+E\+: $\vert$ Software version 1.\+1 \{(@)v(@)\} $\vert$ U\+S\+A\+G\+E\+: $\vert$ The Owl release. \{$\vert$$\sim$-\/ -\/$\sim$$\vert$\} $\vert$ U\+S\+A\+G\+E\+: $\vert$ Copyright (C) Dr. Ivan S Zapreev, 2015-\/2016 \{/$^\wedge$\textquotesingle{}$^\wedge$\textquotesingle{}$^\wedge$ 